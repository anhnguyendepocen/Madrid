\documentclass{beamer}
\usepackage[utf8]{inputenc}
\usepackage{palatino}
\usepackage{subfig}
\usepackage{amsmath}
\usepackage{dsfont}
\usepackage{minted}
\usepackage{multimedia}

\usetheme{Warsaw}
\usecolortheme{crane}

% www.sharelatex.com/learn/Beamer

\title{Parameter Estimation}
\author{Brendon J. Brewer}
\institute{Department of Statistics\\
The University of Auckland}
\date{{\tt \color{blue} https://www.stat.auckland.ac.nz/\~{ }brewer/}}

\begin{document}

\frame{\titlepage}

% New slide
\begin{frame}
\frametitle{Parameter estimation}

Bayesian inference is mostly used for {\em parameter estimation}.
Let $\theta$ be an unknown quantity or `parameter', and $D$ be some data.

\pause

A version of Bayes' rule applies to probability distributions:
\begin{align}
p(\theta | D) &= \frac{p(\theta)p(D|\theta)}{p(D)}
\end{align}

Each term is a distribution, not a single probability.

\end{frame}

% New slide
\begin{frame}
\frametitle{Why?}
Consider three statements {\em about the value of $\theta$}: for example,
$\theta=3.5, \theta=3.6, \theta=3.7$. Consider one statement
{\em about the value of $D$}: for example, $D=5.9$.



\end{frame}


% New slide
\begin{frame}
\frametitle{Why?}

\begin{align}
P(\theta=3.5 | D=5.9) &= \frac{P(\theta=3.5)P(D=5.9 | \theta=3.5)}{P(D=5.9)}\vspace{0.5em} \\
P(\theta=3.6 | D=5.9) &= \frac{P(\theta=3.6)P(D=5.9 | \theta=3.6)}{P(D=5.9)} \\
P(\theta=3.7 | D=5.9) &= \frac{P(\theta=3.7)P(D=5.9 | \theta=3.7)}{P(D=5.9)}
\end{align}

\end{frame}


% New slide
\begin{frame}
\frametitle{Why?}

\begin{align}
{\color{green}P(\theta=3.5 | D=5.9)} &= \frac{{\color{red}P(\theta=3.5)}{\color{blue}P(D=5.9 | \theta=3.5)}}{P(D=5.9)}\vspace{0.5em} \\
{\color{green}P(\theta=3.6 | D=5.9)} &= \frac{{\color{red}P(\theta=3.6)}{\color{blue}P(D=5.9 | \theta=3.6)}}{P(D=5.9)} \\
{\color{green}P(\theta=3.7 | D=5.9)} &= \frac{{\color{red}P(\theta=3.7)}{\color{blue}P(D=5.9 | \theta=3.7)}}{P(D=5.9)}
\end{align}

Green = posterior distribution\\
Red = prior distribution\\
Blue = likelihood function\\
Black = Marginal likelihood / evidence

\end{frame}


% New slide
\begin{frame}
\frametitle{Why?}


\begin{align}
{\color{green}p(\theta | D)} &= \frac{{\color{red}p(\theta)}{\color{blue}p(D|\theta)}}{p(D)}
\end{align}

Green = posterior distribution\\
Red = prior distribution\\
Blue = likelihood function\\
Black = Marginal likelihood / evidence

\end{frame}


% New slide
\begin{frame}
\frametitle{Inputs and outputs}

Inference:\\ (choice of prior, choice of likelihood, the data) $\to$ (posterior, marginal likelihood)

\end{frame}


% New slide
\begin{frame}
\frametitle{Example}

If we knew the intensity of an X-ray source was $\lambda$, we would expect
to see $\lambda t$ photons arrive in a time interval of length $t$.
Let $x_1, x_2, x_3$ be the number of photons observed in three consecutive
minutes.

\begin{align}
x_i | \lambda &\sim \textnormal{Poisson}(\lambda)
\end{align}

i.e.,

\begin{align}
p(x_1, x_2, x_3 | \lambda) &= \prod_{i=1}^3 \frac{\lambda^{x_i} e^{-\lambda x_i}}{x_i!}
\end{align}

Suppose we observed $(x_1, x_2, x_3) = (21, 14, 22)$, and want to infer
$\lambda$.

\end{frame}

% New slide
\begin{frame}
\frametitle{Example}

For Bayes' rule:
\begin{align}
p(\lambda | x_1, x_2, x_3) &= \frac{p(\lambda)p(x_1, x_2, x_3|\lambda)}{p(x_1, x_2, x_3)}
\end{align}

we need a prior, $p(\lambda)$.

\end{frame}


% New slide
\begin{frame}
\frametitle{The log-uniform prior}
How long is a piece of string?

\pause
Twice the distance from the middle to one end.

$\implies$

\begin{align}
p(\lambda) \propto \frac{1}{\lambda}
\end{align}

This is called a log-uniform prior. It `puts the error bars in the exponent'.

\end{frame}


% New slide
\begin{frame}[t, fragile]
\frametitle{Numerical solution with a grid}

\begin{minted}{python}
# Grid of possible parameter values
lamb = np.linspace(0.01, 100.0, 10001)

# Prior density
prior = 1.0 / lamb
prior = prior / np.trapz(prior, x=lamb)

# Likelihood
lik = 
\end{minted}

\end{frame}







\end{document}


