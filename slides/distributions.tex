\documentclass{beamer}
\usepackage[utf8]{inputenc}
\usepackage{palatino}
\usepackage{subfig}
\usepackage{amsmath}
\usepackage{dsfont}
\usepackage{multimedia}

\usetheme{Warsaw}
\usecolortheme{crane}

% www.sharelatex.com/learn/Beamer

\title{Probability Distributions}
\author{Brendon J. Brewer}
\institute{Department of Statistics\\
The University of Auckland}
\date{{\tt \color{blue} https://www.stat.auckland.ac.nz/\~{ }brewer/}}

\begin{document}

\frame{\titlepage}


% New slide
\begin{frame}
\frametitle{Probability Distributions}
Suppose a quantity $X$ might be 1, 2, 3, 4, or 5, and we assign probabilities of
$\frac{1}{5}$ to each of those possible values. There is some terminology:

\begin{itemize}
  \item $X$ is called a `random variable'\footnote{not often by me.}
  \item $\{1, 2, 3, 4, 5\}$ is called the `sample space', `hypothesis space',
        or `parameter space'
  \item $\boldsymbol{p} = \left\{\frac{1}{5}, \frac{1}{5}, \frac{1}{5},
                \frac{1}{5}, \frac{1}{5}\right\}$
        is the `probability distribution' for
        $x$. In this case, it is a {\em discrete uniform} distribution.
\end{itemize}

The probability distribution is often written as $P(X=x) = $(some function of $x$).

\end{frame}



% New slide
\begin{frame}
\frametitle{Properties of discrete probability distributions}

Normalisation:\\ $\sum_x P(X=x) = 1$.\vspace{0.5em}

Expected value:\\$\mathds{E}(X) = \left<X\right> = \sum_x x P(X=x)$\vspace{0.5em}

Variance:\\$\textnormal{Var}(X) = \mathds{E}\left((X - \mathds{E}(X))^2\right) = \sum_x (x - \mathds{E}(X))^2 P(X=x)$\vspace{0.5em}

Standard deviation:\\$\textnormal{sd}(X) = \sqrt{\textnormal{Var}(X)}$

\end{frame}


% New slide
\begin{frame}
\frametitle{Shorthand notation}
$P(X=x)$ is cumbersome. $x$ is also just a dummy variable.

Common shorthand notation: 
Use $p(x)$ instead, equivocate between the quantity itself and the dummy
variable. E.g.:
\begin{align}
\mathds{E}(x) = \sum x p(x)
\end{align}

\end{frame}




\end{document}


