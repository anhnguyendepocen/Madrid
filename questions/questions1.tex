\documentclass[a4paper, 12pt]{article}
\usepackage{amsmath}
\usepackage[utf8]{inputenc}
\usepackage{graphicx}
\usepackage[left=2cm, right=2cm, bottom=3cm, top=2cm]{geometry}
\usepackage{natbib}
\usepackage{microtype}

\newcommand{\given}{\,|\,}

\title{Question Set 1 --- Probability and Bayes}
\author{}
\date{}

\begin{document}
\maketitle

%\abstract{\noindent Abstract}

% Need this after the abstract
\setlength{\parindent}{0pt}
\setlength{\parskip}{8pt}

\section*{Question 1}
Here are four equations relating some probabilities to other probabilities.
\begin{align}
P(A \vee B) &= P(A) + P(B) \\
P(X, Y \given Z) &= P(X \given Z)P(Y \given X, Z) \\
P(a=3, b=4 \given c=5) &= \frac{P(a=3, b=4)P(c = 5 \given a=3, b=4)}
                          {P(c=5)} \\
P(x=3) &= P(x=3 \given y=4)P(y=4) + P(x=3 \given y=5)P(y=5).
\end{align}

For each of the four equations,
\begin{enumerate}
\item Which rule(s) of probability theory the equation is based upon?;
\item Is the equation completely general? If not,
what extra assumptions have been made (e.g. are some of the
propositions independent, or perhaps mutually exclusive?)
\end{enumerate}

\section*{Question 2}

Steve is on trial for a murder. Based on all the evidence presented in court
so far, the jury has narrowed down the set of possibilities to the following
three mutually exclusive hypotheses:\\

\begin{center}
\begin{tabular}{|l|l|l|}
\hline
Hypothesis	&	Definition	  &	Prior Probability\\
\hline
$S$		& {\bf S}teve did it	         & 0.5 \\
$M$		& Steve's brother {\bf M}ike did it & 0.3 \\
$U$     & Someone {\bf U}nrelated to Steve did it & 0.2 \\
\hline
\end{tabular}
\end{center}
Suppose also that Steve wears size 10 shoes all the time,
Mike wears size 10 shoes about
half of the time, and about 10\% of people in general wear men's size 10 shoes.

The murderer's shoeprint is found at the scene, and it is
of a men's size 10 shoe. Let $D$, for data, be the statement that the
shoeprint was a men's size 10.

\begin{enumerate}
\item Find the three likelihoods
$P(D|S)$, $P(D|M)$, an $P(D|U)$\footnote{You'll have to casually equate
a probability to a frequency. That's often okay, but they aren't the
same concept --- they are two separate concepts whose values are being
equated by assumption.}
\item Find the three posterior probabilities $P(S|D)$, $P(S|M)$,
and $P(S|U)$.
\end{enumerate}



\end{document}

