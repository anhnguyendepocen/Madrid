\documentclass[a4paper, 12pt]{article}
\usepackage{amsmath}
\usepackage[utf8]{inputenc}
\usepackage{graphicx}
\usepackage[left=2cm, right=2cm, bottom=3cm, top=2cm]{geometry}
\usepackage{natbib}
\usepackage{microtype}

\newcommand{\given}{\,|\,}

\title{Question Set 1 --- Probability and Bayes}
\author{}
\date{}

\begin{document}
\maketitle

%\abstract{\noindent Abstract}

% Need this after the abstract
\setlength{\parindent}{0pt}
\setlength{\parskip}{8pt}

\section*{Question 1}
Here are four equations relating some probabilities to other probabilities.
\begin{align}
P(A \vee B) &= P(A) + P(B) \\
P(X, Y \given Z) &= P(X \given Z)P(Y \given X, Z) \\
P(a=3, b=4 \given c=5) &= \frac{P(a=3, b=4)P(c = 5 \given a=3, b=4)}
                          {P(c=5)} \\
P(x=3) &= P(x=3 \given y=4)P(y=4) + P(x=3 \given y=5)P(y=5).
\end{align}

For each of the four equations,
\begin{enumerate}
\item Which rule(s) of probability theory the equation is based upon?;
\item Is the equation completely general? If not,
what extra assumptions have been made (e.g. are some of the
propositions independent, or perhaps mutually exclusive?)
\end{enumerate}

\section*{Question 1}
Here are four equations relating some probabilities to other probabilities.
\begin{align}
P(A \vee B) &= P(A) + P(B) \\
P(X, Y \given Z) &= P(X \given Z)P(Y \given X, Z) \\
P(a=3, b=4 \given c=5) &= \frac{P(a=3, b=4)P(c = 5 \given a=3, b=4)}
                          {P(c=5)} \\
P(x=3) &= P(x=3 \given y=4)P(y=4) + P(x=3 \given y=5)P(y=5).
\end{align}

For each of the four equations,
\begin{enumerate}
\item Which rule(s) of probability theory the equation is based upon?;
\item Is the equation completely general? If not,
what extra assumptions have been made (e.g. are some of the
propositions independent, or perhaps mutually exclusive?)
\end{enumerate}


\end{document}

