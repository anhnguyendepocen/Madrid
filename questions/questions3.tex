\documentclass[a4paper, 12pt]{article}
\usepackage{amsmath}
\usepackage[utf8]{inputenc}
\usepackage{graphicx}
\usepackage[left=2cm, right=2cm, bottom=3cm, top=2cm]{geometry}
\usepackage{natbib}
\usepackage{microtype}
\usepackage{hyperref}

\newcommand{\given}{\,|\,}

\title{Question Set 3 --- Parameter Estimation}
\author{}
\date{}

\begin{document}
\maketitle

%\abstract{\noindent Abstract}

% Need this after the abstract
\setlength{\parindent}{0pt}
\setlength{\parskip}{8pt}

\section*{Question 1}
You are a customs agent. Among other things, you are supposed to prevent
drugs being smuggled in packages sent into the country. A colleague finds a
package containing two toys. Let the number of toys containing drugs be
$\eta$. Clearly the value of $\eta$ is either 0, 1, or 2. You drill into
one of the toys and find that it does not have drugs.
Calculate the posterior distribution for $\eta$ given that the tested
toy did not contain drugs. Assume a (discrete) uniform prior, i.e.
$P(\eta=0) = P(\eta=1) = P(\eta=2) = 1/3$.

\section*{Question 2}
An X-ray source emits photons at a steady rate, but since it's so distant, we
only pick up a few photons per second.
A standard probabilistic model for the arrival times of the photons is the
``Poisson process''. A specific prediction of this model is that, if the
expected number of photons in a time interval
is $\lambda$, the (discrete) probability distribution for the actual number of photons $x$ in that interval is a Poisson distribution:
\begin{eqnarray}
p(x | \lambda) &=& \frac{\lambda^x e^{-\lambda}}{x!}
\end{eqnarray}
Find and plot the posterior distribution for $\lambda$ given $x=5$. Use an improper
log-uniform prior proportional to $\lambda^{-1}$. You can do this numerically or
analytically.

\section*{Question 3}
Use the posterior distribution obtained in the previous question to calculate
the predictive distribution for $x'$, the number of photons observed in a
different one second interval, given $x=5$. Plot the predictive distribution
and compare it with what you'd get by naively assuming the best fit
(maximum likelihood) value $\lambda=5$ to make the prediction.\\

{\bf Hint 1: }Write down the probability distribution for $x'$ given
$\lambda$ and $x$, and then marginalise out $\lambda$.\\

{\bf Hint 2: }You may need to do some numerical integration.

\end{document}

