\documentclass[a4paper, 12pt]{article}
\usepackage{amsmath}
\usepackage[utf8]{inputenc}
\usepackage{graphicx}
\usepackage[left=2cm, right=2cm, bottom=3cm, top=2cm]{geometry}
\usepackage{natbib}
\usepackage{microtype}
\usepackage{hyperref}

\newcommand{\given}{\,|\,}

\title{Question Set 2 --- Probability Distributions}
\author{}
\date{}

\begin{document}
\maketitle

%\abstract{\noindent Abstract}

% Need this after the abstract
\setlength{\parindent}{0pt}
\setlength{\parskip}{8pt}

\section*{Question 1}
Photons arrive from a source at a typical rate of $\lambda = 10$ per second.
In any time interval of length $t$, the number $x$ of photons has a
Poisson distribution with parameter $\lambda t$.
Non-overlapping time intervals are independent.

\begin{enumerate}
\item[(a)] Numerically verify that the expected value of $x$ for a one-second
           time interval is 10 and the standard deviation is $\sqrt{10}$.
\item[(b)] Find the probability that there are exactly 10 photons in each of
           two consecutive one-second intervals.
\end{enumerate}


\section*{Question 2}
A ``standard normal'' distribution (with mean zero and standard deviation 1)
has probability density function
\begin{align}
p(x) &= \frac{1}{\sqrt{2\pi}} \exp\left(-\frac{1}{2}x^2\right).
\end{align}

\begin{enumerate}
\item[(a)] Make a grid in Python (or whatever language you like) and plot
           the PDF.
\item[(b)] Numerically calculate $P(x \in [-1, 1])$. You might recognise the
           number as a `one-sigma confidence level'.
\item[(c)] Numerically calculate $P(x \in [-2, 2])$. You might recognise the
           number as a `two-sigma confidence level'.
\item[(d)] Numerically calculate the mean and standard deviation and make sure
           you get something close to 0 and 1 respectively.
\end{enumerate}

A $t$ distribution has a PDF proportional to
\begin{align}
p(x | \nu) &\propto \left(1 + \frac{x^2}{\nu}\right)^{-\frac{\nu + 1}{2}}
\end{align}
The parameter $\nu$ controls the shape of the distribution
(see the plot on the \href{https://en.wikipedia.org/wiki/Student\%27s_t-distribution}{Wikipedia page}, which shows the effect of $\nu$).

\begin{enumerate}
\item[(e)] Let $\nu=3$. Numerically calculate $P(x \in [-2, 2])$.
\end{enumerate}



\end{document}

